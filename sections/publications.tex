\section{Publication (Selected)}
\cventry{2023}{SENSOR: Graph-based Revision History Analysis for Code Evolution
    Introspection}{Akshith Gunasekaran, Huascar Sanchez, Briland Hitaj}{}{} {
    \begin{itemize}
        \item Developed a graph-based representation to analyze the revision
              history of large open source projects to identify code changes
              that introduce security vulnerabilities.
        \item Evaluated the approach on a dataset of over millions of code
              changes and found it to be effective at identifying security
              vulnerabilities.
    \end{itemize}
}

\cventry{2023}{In Pursuit of Lean OS Kernels - Examining Benefits and Barriers
    to Unlocking Aggressive Debloating}{Akshith Gunasekaran, Gabriel Ritter,
    Rakesh Bobba, Yeongjin Jang}{}{} {
    \begin{itemize}
        \item Conducted a systematic study of OS kernel debloating techniques
              and developed a framework for evaluating their effectiveness.
        \item Proposed new composition techniques that enable more aggressive
              debloating and evaluated them on the Linux Kernel.
        \item The proposed techniques have been evaluated on a variety of Linux
              kernels images and have shown to be effective at reducing kernel
              size by at least 20\%.
    \end{itemize}
}

\cventry{2022}
{CONSTRUCT: A Program Synthesis Approach for Reconstructing Control
    Algorithms from Embedded System Binaries in Cyber-Physical Systems} {Ali
    Shokri, Alexandre Perez, Souma Chowdhury, Chen Zeng, Gerald Kaloor, Ion
    Matei, Peter-Patel Schneider, Akshith Gunasekaran, and Shantanu
    Rane}{\href{https://arxiv.org/pdf/2308.00250.pdf}{arXiv:2308.00250}}{} {
    \begin{itemize}
        \item Developed a program synthesis approach to reconstruct control
              algorithms from embedded system binaries.
        \item The approach is based on binary decompilation, static analysis,
              and evolutionary search.
        \item We evaluated on a dataset of real-world embedded systems and was
              able to reconstruct control algorithms with an accuracy of over
              90\%.
        \item The work is being used by researchers at PARC to develop new tools
              for securing and improving the reliability of cyber-physical
              systems.
    \end{itemize}
}

% \cventry{2019}{MultiK: A Framework for Orchestrating Multiple Specialized
% Kernels}{\href{https://arxiv.org/pdf/1903.06889.pdf}{}}{}{}
% {
%     \begin{itemize}
%         \item Operating System Specialization for Performance and Security
%         \item A framework for orchestrating multiple specialized kernels.
%         \item Linux Kernel, Dynamic Analysis, QEMU, GDB.
%     \end{itemize}
% }

% \cventry{NDSS 2019}{Balancing Image Privacy and Usability with
% Thumbnail-Preserving Encryption}{}{}{}
% {
%     \begin{itemize}
%         \item Image encryption scheme to balance privacy and usability.
%         \item Readily Deployable on Existing cloud storage backend.
%         \item Try it at \href{https://photoencryption.org}{photoencryption.org}
%     \end{itemize}
% }
